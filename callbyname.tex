\documentclass[10pt]{beamer}

\usetheme[progressbar=frametitle]{metropolis}
\usepackage{appendixnumberbeamer}
\usepackage[sfdefault]{FiraSans}
\usepackage{FiraMono}
\usepackage{bussproofs}

\usepackage[absolute,overlay]{textpos}
\usepackage{booktabs}
\usepackage[scale=2]{ccicons}



\usepackage{pgfplots}
\usepgfplotslibrary{dateplot}


\usepackage{listings}

\usepackage{tikz}

\usepackage{xspace}
\newcommand{\themename}{\textbf{\textsc{metropolis}}\xspace}
\usepackage{cmll}
\EnableBpAbbreviations

\newcommand{\lam}[2]{\lambda #1 . #2}
\newcommand{\llet}[3]{\text{let }  ! #1 = #2 \text{ in } #3}

\newcommand{\app}[2]{#1 \; #2}
\newcommand{\subst}[2]{[#1 := #2]}
\newcommand{\substt}[4]{[#1 := #2, #3 := #4]}

\newenvironment*{inference}[2]{
  \begin{textblock*}{5cm}(#1,#2)
    \begin{prooftree}

    }
    {
    \end{prooftree}

  \end{textblock*}
}


%%%%%%%%%%%%%%%%%%%%%%%%%%%%
%% UNCC Theme Adjustments %%
%%%%%%%%%%%%%%%%%%%%%%%%%%%%
\definecolor{CanvasBG}{HTML}{FAFAFA}

% From the official style guide
\definecolor{UnccGreen}{HTML}{00703C}
\definecolor{UnccGold}{HTML}{B3A369}
\definecolor{UnccLightGreen}{HTML}{C3D7A4}
\definecolor{UnccYellow}{HTML}{F0CB00}
\definecolor{UnccOrange}{HTML}{F3901D}
\definecolor{UnccLightYellow}{HTML}{FFF6DC}
\definecolor{UnccBlue}{HTML}{00728F}
\definecolor{UnccPink}{HTML}{DE3A6E}
\definecolor{White}{HTML}{FFFFFF}
\definecolor{LightGray}{HTML}{DDDDDD}

% Supporting Color Palette
\definecolor{WarmGray}{HTML}{696158}
\definecolor{StoneGray}{HTML}{717C7D}
\definecolor{DarkGreen}{HTML}{2C5234}
\definecolor{LightGreen}{HTML}{509E2F}
\definecolor{BrightGold}{HTML}{F0CB00}

% Screamers
\definecolor{Royal}{HTML}{72246C}
\definecolor{Ocean}{HTML}{006BA6}
\definecolor{Flash}{HTML}{B52555}
\definecolor{Citrus}{HTML}{FFB81C}
\definecolor{Spring}{HTML}{CEDC00}

% Serenity
\definecolor{Garden}{HTML}{B7CE95}
\definecolor{Sand}{HTML}{F0E991}
\definecolor{Bloom}{HTML}{F1E6B2}
\definecolor{Clay}{HTML}{B7B09C}
\definecolor{Cloud}{HTML}{BAC5B9}

% Set colors here
\setbeamercolor{frametitle}{bg=UnccGreen}
\setbeamercolor{progress bar}{bg=BrightGold, fg=UnccGreen}
\setbeamercolor{alerted text}{fg=Flash}

\setbeamercolor{block title}{bg=LightGreen, fg=White}
\setbeamercolor{block title example}{bg=Ocean, fg=White}
\setbeamercolor{block title alerted}{bg=Citrus, fg=White}
\setbeamercolor{block body}{bg=CanvasBG}

\metroset{titleformat=smallcaps, progressbar=foot}

\makeatletter
\setlength{\metropolis@progressinheadfoot@linewidth}{2pt}
\setlength{\metropolis@titleseparator@linewidth}{2pt}
\setlength{\metropolis@progressonsectionpage@linewidth}{2pt}
%%%%%%%%%%%%%%%%%%%%%%%%%%%%
%% UNCC Theme Adjustments %%
%%%%%%%%%%%%%%%%%%%%%%%%%%%%



\lstset{
  basicstyle=\scriptsize\ttfamily,
  keywordstyle=\color{Ocean},
  stringstyle=,
  xleftmargin=1em,
  showstringspaces=false,
  commentstyle=\itshape\rmfamily,
  columns=flexible,
  keepspaces=true,
  texcl=true
}

% https://www.mathcha.io/editor

\title{Call-by-name, Call-by-value, Call-by-need, and the Linear Lambda Calculus }
% subtitle:
\subtitle{Short Talk}

% \date{\today}
\date{}
\author{Quentin Schroeder}
\institute{MPRI - Université Paris-Cité}
\titlegraphic{\hfill\includegraphics[height=1.5cm]{logo.pdf}}

\begin{document}

\maketitle

% plan
% short motivation
% introduce the lambda calculus and explain the evaluation strategies within it
% introduce the linear lambda calculus
% show how to interpret call by value
% show how to interpret call by name
% shortly talk about the affine lambda calculus
% show how to interpret call by let
% show how to interpret call by need

\section[Idea]{Motivation}

\begin{frame}[fragile]{Motivation}
  \begin{alertblock}{Goal}
    Study evaluation strategies via the linear lambda calculus
  \end{alertblock}

  \begin{alertblock}{Why?}
    \begin{itemize}
      \item found linearity is relevant when studying Call by Need
      \item noticed it also applies for other strategies
    \end{itemize}
  \end{alertblock}
\end{frame}

\begin{frame}[fragile]{Overview}
  \begin{enumerate}
    \item Linear Lambda Calculus
    \item Call by Name
    \item Call by Value
    \item Notes on Call by Need
    \item Results
    \item Conclusion
  \end{enumerate}

\end{frame}



% give me the typing rules for the simply typed lambda calculus in BNF in one line




\begin{frame}[fragile]{Simply Typed Lambda Calculus (Syntax)}
  \begin{textblock*}{3cm}(0.3cm, 1cm)
    \begin{align*}
      \textbf{Types : } A, B , C ::= & \; \text{basic types} \;  | \; A \rightarrow B \\
      \textbf{Terms : } L,M, N ::=   & \; V \; | \; \app{M}{N}                        \\
      \textbf{Values : } V, W ::=    & \; x \; | \; \lam{x}{t}                        \\
    \end{align*}
  \end{textblock*}

  \begin{inference}{0.1cm}{3cm}
    \AXC{}
    \LeftLabel{\scriptsize Id}
    \UIC{$x : A \vdash x : A$}
  \end{inference}

  \begin{inference}{0.1cm}{4cm}
    \AXC{$\Gamma, y : A, z : A \rightarrow M : B$}
    \LeftLabel{\scriptsize Contraction}
    \UIC{$\Gamma , x : A \vdash M \substt{y}{x}{z}{x} : B$}
  \end{inference}

  \begin{inference}{5.5cm}{4cm}
    \AXC{$\Gamma \vdash M : B$}
    \LeftLabel{\scriptsize Weakening}
    \UIC{$\Gamma , x : A \vdash M  : B$}
  \end{inference}

  \begin{inference}{0.1cm}{5cm}
    \AXC{$\Gamma, x:A \vdash M : B$}
    \LeftLabel{\scriptsize $\rightarrow - Intro$}
    \UIC{$\Gamma \vdash \lam{x}{M} : A \rightarrow B$}
  \end{inference}

  \begin{inference}{5.5cm}{5cm}
    \AXC{$\Gamma \vdash M : A \rightarrow B$}
    \AXC{$\Delta \vdash N : A$}
    \LeftLabel{\scriptsize $\rightarrow - Elim$}
    \BIC{$\Gamma, \Delta \vdash \app{M}{N} : B$}
  \end{inference}

\end{frame}

\begin{frame}[fragile]{Simply Typed Lambda Calculus (Evaluation Strategies)}
  \begin{alertblock}{Call by Name}
    Reduces on terms, not values \\
    $(\beta_{name}) : \app{(\lam{x}{M})}{N} \rightsquigarrow M \subst{x}{N}$ \\
  \end{alertblock}

  \begin{alertblock}{Call by Value}
    Reduces on values, not terms \\
    $(\beta_{value}) : \app{(\lam{x}{M})}{V} \rightsquigarrow M \subst{x}{V}$ \\

  \end{alertblock}

\end{frame}

\begin{frame}
  % examples of terms
  \begin{exampleblock}{Example}
    From \cite[SE:101670]{stackexchange:101670}
    \begin{align*}
      \app{\app{(\lam{p}{\lam{q}{p}})}{(\lam{a}{\lam{b}{a}})}}{(\lam{a}{\lam{b}{b}})}
    \end{align*}
  \end{exampleblock}

\end{frame}



\begin{frame}[fragile]{Linear Lambda Calculus (Syntax)}

  \begin{textblock*}{3cm}(0.3cm, 1cm)
    \begin{align*}
      \textbf{Types : } A, B , C ::= & \; \text{basic types} \; | \; ! A \; | \; A \multimap A                         \\
      \textbf{Terms : } L,M, N ::=   & \; x \; | \; ! M  \; | \; \llet{x}{M}{N}  \; | \; \lam{x}{M} \; | \; \app{M}{N} \\
    \end{align*}
  \end{textblock*}

  \begin{inference}{0.1cm}{3cm}
    \AXC{}
    \LeftLabel{\scriptsize Id}
    \UIC{$x : A \vdash x : A$}
  \end{inference}

  \begin{inference}{4cm}{3cm}
    \AXC{$\Gamma, x: A \vdash M : B$}
    \LeftLabel{\scriptsize Dereliction}
    \UIC{$\Gamma, ! x : ! A \vdash M : B$}
  \end{inference}

  \begin{inference}{4cm}{4cm}
    \AXC{$\Gamma, ! y : ! A, ! z : ! A \multimap M : B$}
    \LeftLabel{\scriptsize Contraction}
    \UIC{$\Gamma , ! x : ! A \vdash M \substt{y}{x}{z}{x} : B$}
  \end{inference}

  \begin{inference}{4cm}{4cm}
    \AXC{$\Gamma \vdash M : B$}
    \LeftLabel{\scriptsize Weakening}
    \UIC{$\Gamma , x : A \vdash M  : B$}
  \end{inference}

  \begin{inference}{0.1cm}{4cm}
    \AXC{$! \Gamma \vdash M : A$}
    \LeftLabel{\scriptsize !-Intro}
    \UIC{$! \Gamma \vdash ! M : ! A$}
  \end{inference}

  \begin{inference}{0.1cm}{5cm}
    \AXC{$! \Gamma \vdash M : ! A$}
    \AXC{$\Delta, ! x : ! A \vdash N : B$}
    \LeftLabel{\scriptsize !-Elim}
    \BIC{$\Gamma, \Delta \vdash \llet{x}{M}{N} : B$}
  \end{inference}

\end{frame}








\begin{frame}[fragile]{Preliminaries}
  \begin{alertblock}{Linear Logic}
    \begin{itemize}
      \item a resource sensitive logic
      \item can be used to priority in evaluation of proof terms
    \end{itemize}
  \end{alertblock}

  \pause

  \begin{alertblock}{A massive leap of faith}
    \begin{itemize}
      \item build a linear lambda calculus \cite[]{Maraist1995Jan}
      \item show that it can be used to model execution strategies
    \end{itemize}
  \end{alertblock}
\end{frame}







\begin{frame}[allowframebreaks]{References}
  \bibliographystyle{amsalpha}
  \bibliography{sources}
\end{frame}



\end{document}
