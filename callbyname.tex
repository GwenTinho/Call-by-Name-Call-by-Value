\documentclass[10pt]{beamer}

\usetheme[progressbar=frametitle]{metropolis}
\usepackage{appendixnumberbeamer}

\usepackage{booktabs}
\usepackage[scale=2]{ccicons}

\usepackage{pgfplots}
\usepgfplotslibrary{dateplot}

\usepackage{listings}

\usepackage{tikz}

\usepackage{xspace}
\newcommand{\themename}{\textbf{\textsc{metropolis}}\xspace}

%%% Relevant videos
% https://www.youtube.com/watch?v=NX7D5eEh2n0
%

%%%


%%%%%%%%%%%%%%%%%%%%%%%%%%%%
%% UNCC Theme Adjustments %%
%%%%%%%%%%%%%%%%%%%%%%%%%%%%
\definecolor{CanvasBG}{HTML}{FAFAFA}

% From the official style guide
\definecolor{UnccGreen}{HTML}{00703C}
\definecolor{UnccGold}{HTML}{B3A369}
\definecolor{UnccLightGreen}{HTML}{C3D7A4}
\definecolor{UnccYellow}{HTML}{F0CB00}
\definecolor{UnccOrange}{HTML}{F3901D}
\definecolor{UnccLightYellow}{HTML}{FFF6DC}
\definecolor{UnccBlue}{HTML}{00728F}
\definecolor{UnccPink}{HTML}{DE3A6E}
\definecolor{White}{HTML}{FFFFFF}
\definecolor{LightGray}{HTML}{DDDDDD}

% Supporting Color Palette
\definecolor{WarmGray}{HTML}{696158}
\definecolor{StoneGray}{HTML}{717C7D}
\definecolor{DarkGreen}{HTML}{2C5234}
\definecolor{LightGreen}{HTML}{509E2F}
\definecolor{BrightGold}{HTML}{F0CB00}

% Screamers
\definecolor{Royal}{HTML}{72246C}
\definecolor{Ocean}{HTML}{006BA6}
\definecolor{Flash}{HTML}{B52555}
\definecolor{Citrus}{HTML}{FFB81C}
\definecolor{Spring}{HTML}{CEDC00}

% Serenity
\definecolor{Garden}{HTML}{B7CE95}
\definecolor{Sand}{HTML}{F0E991}
\definecolor{Bloom}{HTML}{F1E6B2}
\definecolor{Clay}{HTML}{B7B09C}
\definecolor{Cloud}{HTML}{BAC5B9}

% Set colors here
\setbeamercolor{frametitle}{bg=UnccGreen}
\setbeamercolor{progress bar}{bg=BrightGold, fg=UnccGreen}
\setbeamercolor{alerted text}{fg=Flash}

\setbeamercolor{block title}{bg=LightGreen, fg=White}
\setbeamercolor{block title example}{bg=Ocean, fg=White}
\setbeamercolor{block title alerted}{bg=Citrus, fg=White}
\setbeamercolor{block body}{bg=CanvasBG}

\metroset{titleformat=smallcaps, progressbar=foot}

\makeatletter
\setlength{\metropolis@progressinheadfoot@linewidth}{2pt}
\setlength{\metropolis@titleseparator@linewidth}{2pt}
\setlength{\metropolis@progressonsectionpage@linewidth}{2pt}
%%%%%%%%%%%%%%%%%%%%%%%%%%%%
%% UNCC Theme Adjustments %%
%%%%%%%%%%%%%%%%%%%%%%%%%%%%



\lstset{
  basicstyle=\scriptsize\ttfamily,
  keywordstyle=\color{Ocean},
  stringstyle=,
  xleftmargin=1em,
  showstringspaces=false,
  commentstyle=\itshape\rmfamily,
  columns=flexible,
  keepspaces=true,
  texcl=true
}
\lstset{escapeinside={<@}{@>}}

% https://www.mathcha.io/editor

\title{Call by Name - Call by Value and the Lambda Calculus}
% \date{\today}
\date{}
\author{Quentin Schroeder}
\institute{MPRI - Université Paris-Cité}
% \titlegraphic{\hfill\includegraphics[height=1.5cm]{logo.pdf}}

\begin{document}

\maketitle

% Outline
% 1. Motivation - why do we care about evaluation strategies
% 2. Preliminaries - introduce the lambda calculus and the two evaluation strategies
% 3. Call by Name - explain the evaluation strategy and its properties
% 4. Call by Name - give examples in the lambda calculus
% 5. Detour: Call by Need - explain the evaluation strategy and its properties
% 6. Detour: Call by Need - give examples in haskell
% 7. Call by Value - explain the evaluation strategy and its properties
% 8. Call by Value - give examples in Ocaml

\section[Idea]{Motivation}

% give two code examples
% one is a place where call by name is better
% one is a place where call by value is better
% explain the difference between the two

% then give code examples in ocaml vs haskell to compare call by need and call by value

\begin{frame}[fragile]{Motivation}
  \begin{alertblock}{Problem}
    We want to understand how to evaluate expressions in the lambda calculus.\\
    We will focus on two strategies: call by name and call by value.\\
  \end{alertblock}

  \pause

  \begin{alertblock}{Example}
    % insert code example of call by value in python

    \begin{lstlisting}[language=Python]
      def f(x):
        return x + x

      print(f(2 + 3))
    \end{lstlisting}



  \end{alertblock}

  \begin{alertblock}{Example }
    % insert code example of call by value in python

    \begin{lstlisting}[language=Python]
      def project(x, y):
        return x

      def loop():
        return loop()

      print(project(2, loop()))
    \end{lstlisting}


  \end{alertblock}

  \begin{alertblock}{Example}
    % insert code example of call by value in python

    \begin{lstlisting}[language=Haskell]
      project x y = x

      loop x = loop x

      main = print (project 2 (loop 3))
    \end{lstlisting}
  \end{alertblock}
\end{frame}


% The problem

\begin{frame}[fragile]{The problem}
  \begin{alertblock}{Problem}
    We want to understand how to evaluate expressions in the lambda calculus.\\
    We will focus on two strategies: call by name and call by value.\\
  \end{alertblock}

  \pause

  \begin{alertblock}{Example}

  \end{alertblock}

  \begin{alertblock}{Question}

  \end{alertblock}
\end{frame}



% overview slide
\begin{frame}[fragile]{Overview}
  \begin{itemize}
    \item Preliminaries
    \item Call by Name
    \item Call by Value
  \end{itemize}
\end{frame}




\begin{frame}[fragile]{The big idea}

  \begin{alertblock}{Idea}
    lorem ipsum
  \end{alertblock}

  \pause

  \begin{alertblock}{Remark}

    lorem ipsum
  \end{alertblock}

\end{frame}



\section{Conclusion}


\begin{frame}[fragile]{Things Left Open}
  \begin{itemize}[<+- | alert@+>]
    \item What blah
  \end{itemize}
\end{frame}



\begin{frame}[standout]
  Thank you
\end{frame}

\begin{frame}[allowframebreaks]{References}
  \bibliographystyle{plainnat}
  \bibliography{sources}
\end{frame}



\end{document}
